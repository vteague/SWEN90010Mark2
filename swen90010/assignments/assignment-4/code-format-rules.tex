\section{Code format rules}
\label{app:code-format-rules}

The layout of code has a strong influence on its readability. Readability is an important characteristic of high integrity software. As such, you are expected to have well-formatted code. 

A code formatting style guide is available at \url{http://en.wikibooks.org/wiki/Ada_Style_Guide/Source_Code_Presentation}. You are free to adopt any guide you wish, or to use your own. However, the following your implementation must adhere to at least the following simple code format rules:

\begin{itemize}[topsep=0mm,itemsep=1mm]

\item Every Ada package must contain a comment at the top of the specification file indicating its purpose.

\item Every function or procedure must contain a comment at the beginning explaining its behaviour. In particular, any assumptions should be clearly stated.

\item Constants and variables must be documented.

\item Variable names must be meaningful.

\item Significant blocks of code must be commented.

However, not every statement in a program needs to be commented. Just as you can write too few comments, it is possible to write too many comments.

\item Program blocks appearing in if-statements, while-loops, etc. must be indented consistently. Tabs or spaces can be used, as long as it is done consistently.

\item Lines must be no longer than 80 characters. You can use the Unix command ``\texttt{wc -L *.ad*}'' to check the maximum length line in your Ada source files.

\end{itemize} 

% LocalWords:  wc adb

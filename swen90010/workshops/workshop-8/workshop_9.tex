\documentclass{article} 

\def\rootdir{../}

\usepackage{graphicx}
\usepackage{times}
\usepackage{multicol}
\usepackage{multirow}
\usepackage{latexsym}
\usepackage{url}
\usepackage{xspace}
\usepackage{verbatim}
\usepackage{booktabs}
\usepackage{fullpage}
\usepackage{fancyvrb}
\usepackage{color}
\usepackage{listings}
\usepackage{amsthm}
\usepackage{amsmath}
\usepackage{enumerate}
\usepackage{subfigure}
\usepackage[sectionbib]{bibunits}
\usepackage{hyperref}
\usepackage{rotating}
\usepackage{Tabbing}
\usepackage[all]{xy}
\usepackage[textsize=scriptsize,textwidth=1cm]{todonotes}

\newlength{\tab}
\setlength{\tab}{1em}
\setlength{\parindent}{0pt}
\setlength{\parskip}{6pt}
\setlength{\evensidemargin}{0.0cm}
\setlength{\oddsidemargin}{0.0cm}
\setlength{\textwidth}{16cm}
%  \setlength{\headsep}{0cm}
\setlength{\headheight}{0cm}
\setlength{\topmargin}{0cm}
\setlength{\textheight}{23cm}
\setlength{\itemsep}{0pt}
\setlength{\topsep}{0pt}


\definecolor{javared}{rgb}{0.6,0,0} % for strings
\definecolor{javagreen}{rgb}{0.25,0.5,0.35} % comments
\definecolor{javapurple}{rgb}{0.5,0,0.35} % keywords
\definecolor{javadocblue}{rgb}{0.25,0.35,0.75} % javadoc
\definecolor{javabackground}{rgb}{0.9,0.9,0.9}
\definecolor{javablack}{rgb}{0,0,0}

\lstset{language=Ada,
  backgroundcolor=\color{javabackground},
  basicstyle=\ttfamily\fontsize{10}{12}\selectfont,
  keywordstyle=\color{javablack}\bfseries,
  aboveskip={1.5\baselineskip},
  stringstyle=\color{javared},
  commentstyle=\color{javagreen},
  morecomment=[s][\color{javadocblue}]{/**}{*/},
  numbers=left,
  numberstyle=\tiny\color{black},
  frame=single,
  numbersep=10pt,
  stepnumber=1,
  tabsize=8,
  xleftmargin=0ex,
  xrightmargin=0ex,
  showspaces=false,
  showstringspaces=false,
  aboveskip=0.5ex
}

\newtheoremstyle{defstyle} % name of the style to be used
  {\parskip} % measure of space to leave above the theorem. E.g.: 3pt
  {\parskip} % measure of space to leave below the theorem. E.g.: 3pt
  {}% name of font to use in the body of the theorem
  {}% measure of space to indent
  {\bf}% name of head font
  {\\}% punctuation between head and body
  {.5em}% space after theorem head; " " = normal interword space
  {\thmname{#1}\thmnumber{ #2}~~ {\rm \thmnote{(#3)}}}% header

\theoremstyle{defstyle}
\newtheorem{mydefinition}{Definition}[chapter]
\newenvironment{definition}{\begin{mydefinition}}{$\Box$\end{mydefinition}}

\newtheorem{myexample}{Example}[chapter]
\newenvironment{example}{\begin{myexample}}{$\Box$\end{myexample}}

\newtheorem{myexercise}{Exercise}[chapter]
\newenvironment{exercise}{\begin{myexercise}}{$\Box$\end{myexercise}}


\algnewcommand\algorithmicskip{\textbf{skip}}
\algnewcommand\algorithmicdone{\textbf{done}}
\algrenewcommand\algorithmicindent{2.0em}%


\def\all{\texttt{\textbf{all}}~}
\def\no{\texttt{\textbf{no}}~}
\def\some{\texttt{\textbf{some}}~}
\def\lone{\texttt{\textbf{lone}}~}
\def\one{\texttt{\textbf{one}}~}
\def\set{\texttt{\textbf{set}}~}
\def\alet{\texttt{\textbf{let}}~}



\begin{document}

\lstset{language=,aboveskip=3mm}
\tutorialtitle{9}

\section*{Introduction}

In the lectures, we discussed at the SPARK language, and discussed some static analysis that is possible using the SPARK examiner. In this workshop, we will use the SPARK examiner and simplifier to prove some important properties about our programs. If these proofs pass, we can guarantee that our source code is free from some fairly serious problems that are permitted in non-safe languages.

This workshop will look at just some of the most important properties.

\section*{The SPARK tools}

In this subject, we will be looking mainly at the SPARK examiner and the GNATProve. These tools prove certain properties about some source code using static analysis. The properties are intended to find errors that will or could possibly result in run-time errors, but find them at design time.

In the first workshop, you would have installed GNAT GPL and SPARK GPL 2014 (unless working on the lab machines where they are already installed). If you did not install the SPARK tools, go back to workshop 1, find the directions, and install them.

\section*{Exercises}

\begin{enumerate}

 \item The source code accompanying the workshop contains a set of SPARK files for use. Create a new project using the source (including the \texttt{workshop.gpr} as a properties file).  Compile the source code in {\tt task1.adb} in the GPS environment.

 Next, run SPARK examiner over the source code in {\tt task1.adb} by going to \emph{Spark} $\rightarrow$ \emph{Examine File} from the GPS menu.

  This reports that the variable {\tt Ok} is used without being initialised, and the   variables {\tt I} and {\tt J} are only initialised if {\tt Ok} is   true. Note the difference between a variable being {\em never} initialised (\texttt{Ok}), and a variable {\em possibly} being uninitialised (\texttt{I} and \texttt{J}).

  Note the differences between the compiler errors/warnings, and the SPARK examiner errors/warnings.

 \item Modify the source code from task 1 to include an {\em    ineffective statement}, which is a statement that can be removed    from a program without changing the behaviour of that program. Run the examiner over the modification, and analyse the results.

  Do these types of problems look familiar? (HINT: Think back to data-flow analysis in SWEN90006!)

 \item Now, open and compile the source code in {\tt task3.adb}, and then run the SPARK examiner and GNATProve over the file. To run GNATProve, select  \emph{Spark} $\rightarrow$ \emph{Prove File}. Click \emph{Execute}.

 Why do you think the proof from \texttt{task3.adb} could not be proved?

  \item Go to the source file {\tt task3.adb} and uncomment the commented lines. Re-run the the SPARK examiner and GNATProve, and see what changes.

 Is the program still not proved? If not, try to change the program to correct this problem. 

HINT: Remember \texttt{Integer'First} \texttt{Integer'Last} return the lowest and highest integers respectively.

 If you are struggling with this task, complete the rest of the workshop and come back to it.


 \item Run the SPARK examiner and GNATProve over the code  {\tt task4.adb}.

  The inability to prove this is actually an indication that the program is not a valid SPARK program (although this is not always the case -- sometimes the tools are just not powerful enough to prove some properties).

  What do you think this failed proof relates to? How could you re-write this to arrive at a correct SPARK program? The relevant types are declared in \texttt{task4.ads}.

 \item Modify line 5 of {\tt task4.adb} from:

\quad\quad {\tt AnArray(AnIndex) := AnArray(AnIndex) + 1;}

  to:

\quad\quad {\tt AnArray(AnIndex) := AnArray(0);}

 Run the SPARK examiner over note the error message. This error will also be generated by the GNAT compiler.



 \item Modify line 5 of {\tt task4.adb} from:

\quad\quad {\tt AnArray(AnIndex) := AnArray(AnIndex) + 1;}

  to:

\quad\quad {\tt AnArray(AnIndex) := AnArray(AnIndex + 1);}

 Run the SPARK examiner over this and see what has failed to prove in {\tt task4/task4procedure.siv}. What do you think this failed proof relates to?

\begin{comment}
 \item The final task is to look at \emph{dead paths}: paths of code that are impossible to execute due to some contradiction in branches. The source is \texttt{task5.ads/b} contains a procedure that takes two Boolean variables, \texttt{A} and \texttt{B}, as input, and outputs a Boolean variable \texttt{C}, which is true if and only if \texttt{A} implies \texttt{B}. That is, if \texttt{A} is false, or \texttt{A} and \texttt{B} is both true. Study the implementation of this in \texttt{task5.adb}.

  Run the SPARK Examiner over \texttt{task5.adb}, but this time, add the switch \texttt{-dpc} to the command line:

   \quad\quad \texttt{spark -vcg -dpc -i=project.idx -config\_file=spark.cfg task5.adb}

 The option \texttt{dpc} stands for ``dead path conjectures'', which means the SPARK Examiner will generator conjectures about dead paths in programs.

  Run \texttt{sparksimp -victor} and \texttt{pogs} again.

  Open \texttt{code.sum} to see if there are any dead paths. \texttt{No DPC} means that there were no related dead path conjectures, while \texttt{Live} and \texttt{Dead} refer to living and dead paths respectively.

  Are there any dead paths? If so, open the file \texttt{task5/task5procedure.zlg} to see what failed to be proved. We can refer to the file \texttt{task5/task5procedure.dpc} to see what the initial conjectures were.

  Fix the implementation of \texttt{Task5Procedure} to eliminate the dead path.
\end{comment}
\end{enumerate}

\end{document}

% LocalWords:  gnatmake sparksimp AnArray AnIndex undischarged dpc
% LocalWords:  aboveskip uncomment simplifier GNATProve GPL GPS SWEN
% LocalWords:  config cfg

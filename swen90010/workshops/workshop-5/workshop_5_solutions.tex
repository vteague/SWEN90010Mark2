\documentclass{article} 

\def\rootdir{..}

\usepackage{graphicx}
\usepackage{times}
\usepackage{multicol}
\usepackage{multirow}
\usepackage{latexsym}
\usepackage{url}
\usepackage{xspace}
\usepackage{verbatim}
\usepackage{booktabs}
\usepackage{fullpage}
\usepackage{fancyvrb}
\usepackage{color}
\usepackage{listings}
\usepackage{amsthm}
\usepackage{amsmath}
\usepackage{enumerate}
\usepackage{subfigure}
\usepackage[sectionbib]{bibunits}
\usepackage{hyperref}
\usepackage{rotating}
\usepackage{Tabbing}
\usepackage[all]{xy}
\usepackage[textsize=scriptsize,textwidth=1cm]{todonotes}

\newlength{\tab}
\setlength{\tab}{1em}
\setlength{\parindent}{0pt}
\setlength{\parskip}{6pt}
\setlength{\evensidemargin}{0.0cm}
\setlength{\oddsidemargin}{0.0cm}
\setlength{\textwidth}{16cm}
%  \setlength{\headsep}{0cm}
\setlength{\headheight}{0cm}
\setlength{\topmargin}{0cm}
\setlength{\textheight}{23cm}
\setlength{\itemsep}{0pt}
\setlength{\topsep}{0pt}


\definecolor{javared}{rgb}{0.6,0,0} % for strings
\definecolor{javagreen}{rgb}{0.25,0.5,0.35} % comments
\definecolor{javapurple}{rgb}{0.5,0,0.35} % keywords
\definecolor{javadocblue}{rgb}{0.25,0.35,0.75} % javadoc
\definecolor{javabackground}{rgb}{0.9,0.9,0.9}
\definecolor{javablack}{rgb}{0,0,0}

\lstset{language=Ada,
  backgroundcolor=\color{javabackground},
  basicstyle=\ttfamily\fontsize{10}{12}\selectfont,
  keywordstyle=\color{javablack}\bfseries,
  aboveskip={1.5\baselineskip},
  stringstyle=\color{javared},
  commentstyle=\color{javagreen},
  morecomment=[s][\color{javadocblue}]{/**}{*/},
  numbers=left,
  numberstyle=\tiny\color{black},
  frame=single,
  numbersep=10pt,
  stepnumber=1,
  tabsize=8,
  xleftmargin=0ex,
  xrightmargin=0ex,
  showspaces=false,
  showstringspaces=false,
  aboveskip=0.5ex
}

\newtheoremstyle{defstyle} % name of the style to be used
  {\parskip} % measure of space to leave above the theorem. E.g.: 3pt
  {\parskip} % measure of space to leave below the theorem. E.g.: 3pt
  {}% name of font to use in the body of the theorem
  {}% measure of space to indent
  {\bf}% name of head font
  {\\}% punctuation between head and body
  {.5em}% space after theorem head; " " = normal interword space
  {\thmname{#1}\thmnumber{ #2}~~ {\rm \thmnote{(#3)}}}% header

\theoremstyle{defstyle}
\newtheorem{mydefinition}{Definition}[chapter]
\newenvironment{definition}{\begin{mydefinition}}{$\Box$\end{mydefinition}}

\newtheorem{myexample}{Example}[chapter]
\newenvironment{example}{\begin{myexample}}{$\Box$\end{myexample}}

\newtheorem{myexercise}{Exercise}[chapter]
\newenvironment{exercise}{\begin{myexercise}}{$\Box$\end{myexercise}}


\algnewcommand\algorithmicskip{\textbf{skip}}
\algnewcommand\algorithmicdone{\textbf{done}}
\algrenewcommand\algorithmicindent{2.0em}%


\def\all{\texttt{\textbf{all}}~}
\def\no{\texttt{\textbf{no}}~}
\def\some{\texttt{\textbf{some}}~}
\def\lone{\texttt{\textbf{lone}}~}
\def\one{\texttt{\textbf{one}}~}
\def\set{\texttt{\textbf{set}}~}
\def\alet{\texttt{\textbf{let}}~}


\begin{document}

%----------------------------------------------------------------------
%    Title Page
%

\tutorialsolutionstitle{5}
\lstset{language=}
\section*{Sample solutions for tasks}

The Sum specification and script for executing the answers below are available in the subject repository.

\begin{enumerate}

 \item
\lstinputlisting[linerange={1-9}]{\rootdir/workshop-5/models/storage_solution_clean.script}

 \item 
\lstinputlisting[linerange={11-15}]{\rootdir/workshop-5/models/storage_solution_clean.script}

 \item 
\lstinputlisting[linerange={17-21}]{\rootdir/workshop-5/models/storage_solution_clean.script}

 \item 
\lstinputlisting[linerange={23-24}]{\rootdir/workshop-5/models/storage_solution_clean.script}

Possum has responded with ``no solution''. What Possum tries to do is find bindings for all variables in the operation as it is executed. However, if we try bind \texttt{-1} to the variable \texttt{amount?}, this will fail, because \texttt{amount? : nat} declares that the variable \texttt{amount?} can only take on values from 0 upwards. As such, Possum cannot find a solution to the query \texttt{Fill \{-1/amount?\}} because one does not exist.

The ``no solution'' response will become familiar if you use Possum on assignment 3. It always means that Possum was unable to find a solution (but not necessarily that one does not exist). It can be caused by having a precondition or postcondition that is inconsistent, or from simple problems with variable declarations.

For those of you familiar with logic programming, it is the same as being unable to find bindings for variables in a goal. In fact, Possum is implemented in Mercury -- a logic programming language conceived and implemented at the University of Melbourne.
 
 \item This operation has been divided into its normal and exceptional behaviours using the schema calculus:

\lstinputlisting[linerange={72-93}]{\rootdir/workshop-5/models/storage_workshop_5_solution.sum}

We run a couple of tests to convince ourselves that this is working for a couple of basic cases on the boundary:

\lstinputlisting[linerange={26-36}]{\rootdir/workshop-5/models/storage_solution_clean.script}


\item The \emph{fillsafe} axiom is specified as follows:

\lstinputlisting[linerange={3-10}]{\rootdir/workshop-5/models/storage_workshop_5_solution.sum}

We update the \texttt{FillOK} and \texttt{OverFill} operations to the following:


\lstinputlisting[linerange={47-68}]{\rootdir/workshop-5/models/storage_workshop_5_solution.sum}

Again, we run a couple of tests to check our solution:


\lstinputlisting[linerange={38-48}]{\rootdir/workshop-5/models/storage_solution_clean.script}

\end{enumerate}

\end{document}

% LocalWords:  nat fillsafe FillOK OverFill

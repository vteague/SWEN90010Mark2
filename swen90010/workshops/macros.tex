\newtheoremstyle{defstyle} % name of the style to be used
  {\parskip} % measure of space to leave above the theorem. E.g.: 3pt
  {\parskip} % measure of space to leave below the theorem. E.g.: 3pt
  {}% name of font to use in the body of the theorem
  {}% measure of space to indent
  {\bf}% name of head font
  {\\}% punctuation between head and body
  {.5em}% space after theorem head; " " = normal interword space
  {\thmname{#1}\thmnumber{ #2}~~ {\rm \thmnote{(#3)}}}% header

\newcommand{\tutorialtitle}[1]{
\begin{center}
\textbf{\sc SWEN90010: High Integrity Systems Engineering}\\[0.5ex]
\textbf{\sc Department of Computing and Information Systems}\\[0.5ex]
\textbf{\sc The University of Melbourne}\\[2ex]
\textbf{\large Workshop {#1}}
\end{center}
}


\newcommand{\tutorialsolutionstitle}[1]{
\begin{center}
\textbf{\sc SWEN90010: High Integrity Systems Engineering}\\[0.5ex]
\textbf{\sc Department of Computing and Information Systems}\\[0.5ex]
\textbf{\sc The University of Melbourne}\\[2ex]
\textbf{\large Workshop {#1} sample solutions}
\end{center}
}

\newcommand{\tobedone}[1]{\todo[inline,color=green!20!white]{\textbf{TODO:} #1}}

\newcommand{\tm}[1]{\todo[inline,color=yellow]{\textbf{Tim says:} #1}}

\documentclass[11pt]{article}

\usepackage{fullpage}

\def\rootdir{../}

\usepackage{graphicx}
\usepackage{times}
\usepackage{multicol}
\usepackage{multirow}
\usepackage{latexsym}
\usepackage{url}
\usepackage{xspace}
\usepackage{verbatim}
\usepackage{booktabs}
\usepackage{fullpage}
\usepackage{fancyvrb}
\usepackage{color}
\usepackage{listings}
\usepackage{amsthm}
\usepackage{amsmath}
\usepackage{enumerate}
\usepackage{subfigure}
\usepackage[sectionbib]{bibunits}
\usepackage{hyperref}
\usepackage{rotating}
\usepackage{Tabbing}
\usepackage[all]{xy}
\usepackage[textsize=scriptsize,textwidth=1cm]{todonotes}

\newlength{\tab}
\setlength{\tab}{1em}
\setlength{\parindent}{0pt}
\setlength{\parskip}{6pt}
\setlength{\evensidemargin}{0.0cm}
\setlength{\oddsidemargin}{0.0cm}
\setlength{\textwidth}{16cm}
%  \setlength{\headsep}{0cm}
\setlength{\headheight}{0cm}
\setlength{\topmargin}{0cm}
\setlength{\textheight}{23cm}
\setlength{\itemsep}{0pt}
\setlength{\topsep}{0pt}


\definecolor{javared}{rgb}{0.6,0,0} % for strings
\definecolor{javagreen}{rgb}{0.25,0.5,0.35} % comments
\definecolor{javapurple}{rgb}{0.5,0,0.35} % keywords
\definecolor{javadocblue}{rgb}{0.25,0.35,0.75} % javadoc
\definecolor{javabackground}{rgb}{0.9,0.9,0.9}
\definecolor{javablack}{rgb}{0,0,0}

\lstset{language=Ada,
  backgroundcolor=\color{javabackground},
  basicstyle=\ttfamily\fontsize{10}{12}\selectfont,
  keywordstyle=\color{javablack}\bfseries,
  aboveskip={1.5\baselineskip},
  stringstyle=\color{javared},
  commentstyle=\color{javagreen},
  morecomment=[s][\color{javadocblue}]{/**}{*/},
  numbers=left,
  numberstyle=\tiny\color{black},
  frame=single,
  numbersep=10pt,
  stepnumber=1,
  tabsize=8,
  xleftmargin=0ex,
  xrightmargin=0ex,
  showspaces=false,
  showstringspaces=false,
  aboveskip=0.5ex
}

\newtheoremstyle{defstyle} % name of the style to be used
  {\parskip} % measure of space to leave above the theorem. E.g.: 3pt
  {\parskip} % measure of space to leave below the theorem. E.g.: 3pt
  {}% name of font to use in the body of the theorem
  {}% measure of space to indent
  {\bf}% name of head font
  {\\}% punctuation between head and body
  {.5em}% space after theorem head; " " = normal interword space
  {\thmname{#1}\thmnumber{ #2}~~ {\rm \thmnote{(#3)}}}% header

\newcommand{\tutorialtitle}[1]{
\begin{center}
\textbf{\sc SWEN90010: High Integrity Systems Engineering}\\[0.5ex]
\textbf{\sc Department of Computing and Information Systems}\\[0.5ex]
\textbf{\sc The University of Melbourne}\\[2ex]
\textbf{\large Workshop {#1}}
\end{center}
}


\newcommand{\tutorialsolutionstitle}[1]{
\begin{center}
\textbf{\sc SWEN90010: High Integrity Systems Engineering}\\[0.5ex]
\textbf{\sc Department of Computing and Information Systems}\\[0.5ex]
\textbf{\sc The University of Melbourne}\\[2ex]
\textbf{\large Workshop {#1} sample solutions}
\end{center}
}

\newcommand{\tobedone}[1]{\todo[inline,color=green!20!white]{\textbf{TODO:} #1}}

\newcommand{\tm}[1]{\todo[inline,color=yellow]{\textbf{Tim says:} #1}}


\newlength{\tab}
\setlength{\tab}{1em}
\setlength{\parindent}{0pt}
\setlength{\parskip}{6pt}
\setlength{\evensidemargin}{0.0cm}
\setlength{\oddsidemargin}{0.0cm}
\setlength{\textwidth}{16cm}
%  \setlength{\headsep}{0cm}
\setlength{\headheight}{0cm}
\setlength{\topmargin}{0cm}
\setlength{\textheight}{23cm}
\setlength{\itemsep}{0pt}
\setlength{\topsep}{0pt}


\definecolor{javared}{rgb}{0.6,0,0} % for strings
\definecolor{javagreen}{rgb}{0.25,0.5,0.35} % comments
\definecolor{javapurple}{rgb}{0.5,0,0.35} % keywords
\definecolor{javadocblue}{rgb}{0.25,0.35,0.75} % javadoc
\definecolor{javabackground}{rgb}{0.9,0.9,0.9}
\definecolor{javablack}{rgb}{0,0,0}

\lstset{language=Ada,
  backgroundcolor=\color{javabackground},
  basicstyle=\ttfamily\fontsize{10}{12}\selectfont,
  keywordstyle=\color{javablack}\bfseries,
  aboveskip={1.5\baselineskip},
  stringstyle=\color{javared},
  commentstyle=\color{javagreen},
  morecomment=[s][\color{javadocblue}]{/**}{*/},
  numbers=left,
  numberstyle=\tiny\color{black},
  frame=single,
  numbersep=10pt,
  stepnumber=1,
  tabsize=8,
  xleftmargin=0ex,
  xrightmargin=0ex,
  showspaces=false,
  showstringspaces=false,
  aboveskip=0.5ex
}


\begin{document}

\section*{SWEN90010 lecture -- Ada}

\section*{Lecture outline}

 \begin{enumerate}
  
  \item Obtain an initial understanding of the basic Ada language -- enough to construct a small program.

  \item To understand the properties of Ada that make it suitable for high integrity systems.


 \end{enumerate}

\section*{Lecture plan}

\begin{enumerate}

 \item History:

  \begin{enumerate}

   \item Introduced in 1977 when DoD became frustrated with supporting \emph{hundreds} of programming languages (platform specific, obsolete, not re-usable). Wanted a \emph{safe, modular} programming language.

   \item Four contractors asked to put forward designs;  CII Honeywell Bull's proposal called \emph{Ada} was selected. 

   \item Designed for embedded, real-time systems, but adopted more generally.

   \item Named after \emph{Augusta Ada King, Countess of Lovelace}, (more generally known as \emph{Ada Lovelace}) -- thought to be the first computer programmer on Charles Babbage's \emph{analytical engine} (calculating Bernoulli numbers).

  \end{enumerate}

\item Ada for software engineering:

   \begin{enumerate}

   \item  A strong, static and safe type system; all type errors detected automatically at compile time.

  \item Modularity. Ada modules (called ``packages'') compiled separately before implementation.

  \item Information hiding. 

  \item Readability. Ada favours the \emph{reader} over the \emph{writer}.

  \item Portability. 
  
  \item Standardisation. Ada is an ANSI and ISO standard.

   \end{enumerate}

 \item hello\_world.adb:  compilation, procedure, ``with'', ``begin..end'', must be called ``hello\_world.adb''

 \item greetings.ads/b:  package specifications and bodies.

 \item Types: major reason for Ada's success. Of important: types are equivalent by \emph{name only}.

  \begin{enumerate}

  \item Integer, Float, Character, String, Boolean, etc.

  \item Arrays: indices are not (just) integers (array\_examples.adb)

  \item New data types: dates.ads

  \end{enumerate}

  \item Control structures: if-then, case, loops,

  \item Procedures and functions.

   A procedure call is a statement and does not return any value and is considered a statement. 

   A function returns a value and is itself an expression. 

  \item Parameter modes: in, out, in out, access (pointers)

  \item Calling: no parameters, positional, named associations.

  \item Tasks:

   \begin{enumerate}
     \item Show the ``housekeeping'' template, fill in the body, and run the tasks. Note the ``null'' main procedure.

    \item Show the ``entry call'' example (without the \texttt{terminate}), which explains entry calls. Demonstrate the lack of termination. Add in termination condition to the select statement.
    \item Insert the guard \texttt{Datum < 100} into the entry call example, and modify the main procedure to set \texttt{Datum} to be greater than 100. Re-run the example.
   \end{enumerate}

 \item Subtyping: If time permits, lead into workshop 2 with an example of subtyping in \texttt{subtyping.adb}. Play around with the two sub-types, and show how the compiler detects out of bounds errors, but does not specify that the types are incompatible:

   Example of why subtyping is good: Mars Climate Orbiter. Disintegrated in 1999 when going towards Mars. Too close to the planet due to the ground-based software outputing non-SI pounds per second instead of the metric Netwons per second.

\end{enumerate}

\pagebreak

\lstinputlisting[caption={hello\_world.adb}]{./code/hello_world.adb}

\vfill

\lstinputlisting[caption={greetings.ads}]{./code/greetings.ads}

\lstinputlisting[caption={greetings.adb}]{./code/greetings.adb}

\pagebreak

\lstinputlisting[caption={array\_examples.adb}]{./code/array_examples.adb}

\vfill

\lstinputlisting[caption={date.ads}]{./code/date.ads}

\pagebreak

\lstinputlisting[caption={calling\_subprograms.adb}]{./code/calling_subprograms.adb}

\vfill

\lstinputlisting[caption={subtyping.adb}]{./code/subtyping.adb}


\end{document}  

% LocalWords:  mins atomicity DoD CII adb Subtyping subtyping

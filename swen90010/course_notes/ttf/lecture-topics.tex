\documentclass[11pt]{article}

\def\rootdir{../}
\usepackage{graphicx}
\usepackage{times}
\usepackage{multicol}
\usepackage{multirow}
\usepackage{latexsym}
\usepackage{url}
\usepackage{xspace}
\usepackage{verbatim}
\usepackage{booktabs}
\usepackage{fullpage}
\usepackage{fancyvrb}
\usepackage{color}
\usepackage{listings}
\usepackage{amsthm}
\usepackage{amsmath}
\usepackage{enumerate}
\usepackage{subfigure}
\usepackage[sectionbib]{bibunits}
\usepackage{hyperref}
\usepackage{rotating}
\usepackage{Tabbing}
\usepackage[all]{xy}
\usepackage[textsize=scriptsize,textwidth=1cm]{todonotes}

\newlength{\tab}
\setlength{\tab}{1em}
\setlength{\parindent}{0pt}
\setlength{\parskip}{6pt}
\setlength{\evensidemargin}{0.0cm}
\setlength{\oddsidemargin}{0.0cm}
\setlength{\textwidth}{16cm}
%  \setlength{\headsep}{0cm}
\setlength{\headheight}{0cm}
\setlength{\topmargin}{0cm}
\setlength{\textheight}{23cm}
\setlength{\itemsep}{0pt}
\setlength{\topsep}{0pt}


\definecolor{javared}{rgb}{0.6,0,0} % for strings
\definecolor{javagreen}{rgb}{0.25,0.5,0.35} % comments
\definecolor{javapurple}{rgb}{0.5,0,0.35} % keywords
\definecolor{javadocblue}{rgb}{0.25,0.35,0.75} % javadoc
\definecolor{javabackground}{rgb}{0.9,0.9,0.9}
\definecolor{javablack}{rgb}{0,0,0}

\lstset{language=Ada,
  backgroundcolor=\color{javabackground},
  basicstyle=\ttfamily\fontsize{10}{12}\selectfont,
  keywordstyle=\color{javablack}\bfseries,
  aboveskip={1.5\baselineskip},
  stringstyle=\color{javared},
  commentstyle=\color{javagreen},
  morecomment=[s][\color{javadocblue}]{/**}{*/},
  numbers=left,
  numberstyle=\tiny\color{black},
  frame=single,
  numbersep=10pt,
  stepnumber=1,
  tabsize=8,
  xleftmargin=0ex,
  xrightmargin=0ex,
  showspaces=false,
  showstringspaces=false,
  aboveskip=0.5ex
}

\newtheoremstyle{defstyle} % name of the style to be used
  {\parskip} % measure of space to leave above the theorem. E.g.: 3pt
  {\parskip} % measure of space to leave below the theorem. E.g.: 3pt
  {}% name of font to use in the body of the theorem
  {}% measure of space to indent
  {\bf}% name of head font
  {\\}% punctuation between head and body
  {.5em}% space after theorem head; " " = normal interword space
  {\thmname{#1}\thmnumber{ #2}~~ {\rm \thmnote{(#3)}}}% header

\newcommand{\tutorialtitle}[1]{
\begin{center}
\textbf{\sc SWEN90010: High Integrity Systems Engineering}\\[0.5ex]
\textbf{\sc Department of Computing and Information Systems}\\[0.5ex]
\textbf{\sc The University of Melbourne}\\[2ex]
\textbf{\large Workshop {#1}}
\end{center}
}


\newcommand{\tutorialsolutionstitle}[1]{
\begin{center}
\textbf{\sc SWEN90010: High Integrity Systems Engineering}\\[0.5ex]
\textbf{\sc Department of Computing and Information Systems}\\[0.5ex]
\textbf{\sc The University of Melbourne}\\[2ex]
\textbf{\large Workshop {#1} sample solutions}
\end{center}
}

\newcommand{\tobedone}[1]{\todo[inline,color=green!20!white]{\textbf{TODO:} #1}}

\newcommand{\tm}[1]{\todo[inline,color=yellow]{\textbf{Tim says:} #1}}



\begin{document}

\section*{Lecture aims}

 \begin{enumerate}


 \item Systematically construct tests from formal models using the Test Template Framework.

 \item Apply adapted Test Template Framework to Alloy models.

\end{enumerate}

\section*{Lecture plan}
  
\subsubsection*{Introduction to model-based testing}

\begin{enumerate}

 \item Why test in high-integrity systems? Catch mistakes. Test in environment. Validation.

 \item Testing can be: ad-hoc, exploratory, or systematic. For high-integrity systems, systematic is essential, exploratory is useful, ad-hoc is almost worthless.

 \item Model-based testing: use of design models to generate test artifacts (inputs, oracles, stubs, drivers).

 \item Example: triangle program. Show Alloy definition (below)

\end{enumerate}

\subsubsection*{Test template framework}

\begin{enumerate}

 \item Early model-based testing, which has inspired many new tools and frameworks.

 \item It considers testing to be more than just a statement about input data --- the functional behaviour specification, the test outputs, and test purpose are all important considerations that must be taken into account.

 \item Process: (1) IS, OS, VIS; (2) partition; (3) prune; (4) select inputs; and (5) derive outputs.

 \item Show IS, VIS, and OS for the Triangle program.

 \item Partitioning tactics:

  \begin{enumerate}

   \item Cause-effect analysis: four types of triangle (described using input space)

   \item Partition analysis: 

\lstset{aboveskip=3mm}
\lstset{language=alloy}
\begin{lstlisting}
 PA_ISO.1 == [CE_ISO | x = y and z != x]
 PA_ISO.2 == [CE_ISO | x = z and y != x]
 PA_ISO.3 == [CE_ISO | y = z and y != x]
\end{lstlisting}

   \item Ordered types permutation:

\lstset{aboveskip=3mm}
\lstset{language=alloy}
\begin{lstlisting}
 PERM_SCA.xyz == [CE_SCA | x < y < z]
 PERM_SCA.xzy == [CE_SCA | x < z < y]
 PERM_SCA.yxz == [CE_SCA | y < x < z]
 PERM_SCA.yzx == [CE_SCA | y < z < x]
 PERM_SCA.zyx == [CE_SCA | z < y < x]
 PERM_SCA.zxy == [CE_SCA | z < x < y]
\end{lstlisting}

   \item Boundary value analysis: e.g. divide \verb+x <= y+ into cases \verb+x < y+ and \verb+x=y+.

  \end{enumerate}

  \item Test template hierarchies: draw for the simple triangle example so far. Note that if all children are disjoint and form the parent that: (1) all leaf nodes are disjoint; and (2) their disjunction/union is the VIS!

  \item Test template: path from VIS to leaf; or expanded schema.

  \item Prune infeasible paths. Explain and show an example of doing this with Alloy.

  \item Generate inputs and expected outputs using Alloy. Show with \texttt{PERM\_SCA.xyz} example.

 \item Recap the process.

 \item Discuss automation. The example is small but tedious. Tools/frameworks: Fastest (Flowgate consulting), Smartesting, SpecExplorer (Microsoft).
\end{enumerate}


\pagebreak

\lstset{language=alloy}
\alloyfile{language=alloy}{\rootdir/ttf/models/triangle.als}


\end{document}  


% LocalWords:  hoc VIS aboveskip SCA xyz xzy yxz yzx zyx zxy
% LocalWords:  lll Flowgate Smartesting SpecExplorer EQI INV dec
% LocalWords:  validTriangle

\chapter*{Aims, assessment, and course outline}

\section*{Subject Overview}

The primary focus of this subject is software design and architecture for enterprise systems. The principles in this subject are not restricted to enterprise systems --- they can be applied to other types of systems --- however, most of the topics are taken from areas that are commonly found in enterprise systems engineering.

The topics in this subject include:

\begin{description}

\item[An introduction to enterprise systems] The first part of the subject is an overview of enterprise systems, and introduces general methods and concepts found in enterprise software architecture and design.

\item[Design principles and patterns] This is the main part of the subject. In this section, we will look at common design principles and patterns for enterprise system components,  principles and patterns for bringing together these components, and the trade-offs to be made in these designs.

\item[Design for non-functional requirements] This part of the subject will look at design principles and patterns for non-functional requirements. We will look at only a handful of non-functional requirements, in order to show ways in which design patterns and principles can be applied to non-functional requirements, and how the design can be traced to these non-functional requirements.

\item[Enterprise system integration] The final part of the subject will look at design principles and patterns for integrating different enterprise systems with each other. This is a particular challenging task because different enterprise systems are typically built at different times using different technologies, but will have to work together.

\end{description}

\section*{Staff details}

\begin{tabular}{p{0.49\textwidth}p{0.49\textwidth}}
Shanika Karunasekera & Tim Miller\\
Room: 7.17, Doug McDonell Building & Room: 6.09, Doug McDonell Building \\
Email: \texttt{karus@unimelb.edu.au} & Email: \texttt{tmiller@unimelb.edu.au}\\
\end{tabular}


\section*{Learning outcomes}

On completion of this subject, students should be able to:

\begin{itemize}

  \item Analyse large scale and distributed enterprise systems to select appropriate architectures for them.

  \item Evaluate architectural designs for enterprise systems.

 \item Make suitable trade-offs between different architectures.

\end{itemize}

	
This subject also aims to reinforce the following generic skills:

\begin{itemize}

 \item Ability to apply knowledge of science and engineering fundamentals.

 \item Ability to undertake problem identification, formulation, and solution.

 \item Ability to utilise a systems approach to complex problems and to design and operational performance.

 \item Proficiency in engineering design.
 
 \item Ability to manage information and documentation.

  \item Capacity for creativity and innovation.

  \item Ability to function effectively as an individual and in multidisciplinary and multicultural teams, as a team leader or manager as well as an effective team member.

 \item Capacity for lifelong learning and professional development.

\end{itemize}


\section*{Assumed background}

\begin{itemize}

 \item A solid understanding of the ``Gang of Four'' design patterns (SWEN30006).

 \item Ability to design a medium-scale system.

 \item Ability to apply UML to the documentation of designs.

 \item Practical experience with Java and related tools.

\end{itemize}

\section*{Assessment}

This subject, like all software engineering subjects at the University of Melbourne, is a blend of theory and practice. The assessment has two parts:

\begin{itemize}

 \item {\em Project:} One project, broken into three parts, totalling 30\%:

 \item {\em Exam:} A two-hour end-of-semester examination, worth 70\%.

\end{itemize}

\subsection*{Summary of the subject's assessment}

\vspace{2ex}
\begin{center}
\begin{tabular}{lll}
 \textbf{Assessment} & \textbf{Percentage} & \\
 \textbf{Component}  & \textbf{of Total}   & \textbf{Hurdle} \\
 \hline\hline
 Project                       & 30\%                        & 15\%\\
 Final Examination               & 70\%                        & 35\% \\
\hline
\end{tabular}
\end{center}
\vspace{2ex}

\section*{Contact hours}

\begin{itemize}

 \item Lectures: 2 hours, back-to-back, on Tuesday afternoons, 14:15--16:15.

 \item Tutorial/workshop: 1 hour on Wednesdays at either 9:00--10:00, 10:00-11:00, or 12:00-13:00.

\end{itemize}

It is expected that you will attend lectures and workshops regularly. Lectures will be made available via Lectopia, but given the lecturing style of the subject coordinators, these may not be all that useful. 

\section*{Resources}

The main resource used for communicating between staff and students is the Learning Management System (LMS): \url{http://www.lms.unimelb.edu.au/}. The project, notes, and workshops will be made available on the LMS. Announcements will be put up on the LMS.

A discussion board is available on the LMS. All questions relating to the subject, except those related to personal issues, such as project solutions, should be posted to the LMS discussion board.

\section*{Detailed topics list}

\begin{enumerate}[A.]

\item Introduction 

 1. Introduction to enterprise systems

\item Design principles and patterns for enterprise systems

 2. Organising domain logic

 3. Mapping to the data source layer

 4. The presentation layer

 5. Sessions

 6. Concurrency

 7. Distribution

\item Design principles and patterns for non-functional requirements

 8. Security

 9. Performance

\item Enterprise system integration

\end{enumerate}

%%  LocalWords:  Shanika Karunasekera McDonell lll Lectopia

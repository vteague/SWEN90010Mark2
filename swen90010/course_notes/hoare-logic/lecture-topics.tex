\documentclass[11pt]{article}

\def\rootdir{../}

\usepackage{fullpage}
\usepackage{booktabs}
\usepackage{listings}
\usepackage{url}
\usepackage{oz}
\usepackage{algorithm}
\usepackage{algpseudocode}

\algnewcommand\algorithmicskip{\textbf{skip}}
\algnewcommand\algorithmicdone{\textbf{done}}
\usepackage{color}

\definecolor{javared}{rgb}{0.6,0,0} % for strings
\definecolor{javagreen}{rgb}{0.25,0.5,0.35} % comments
\definecolor{javapurple}{rgb}{0.5,0,0.35} % keywords
\definecolor{javadocblue}{rgb}{0.25,0.35,0.75} % javadoc
\definecolor{javabackground}{rgb}{0.9,0.9,0.9}
\definecolor{javablack}{rgb}{0,0,0}


\lstset{language=Java,
  backgroundcolor=\color{javabackground},
  basicstyle=\ttfamily\fontsize{10}{12}\selectfont,
  keywordstyle=\color{javablack}\bfseries,
  aboveskip={1.5\baselineskip},
  stringstyle=\color{javared},
  commentstyle=\color{javagreen},
  morecomment=[s][\color{javadocblue}]{/**}{*/},
  numbers=left,
  numberstyle=\tiny\color{black},
  frame=single,
  numbersep=10pt,
  stepnumber=1,
  tabsize=8,
  xleftmargin=0ex,
  xrightmargin=0ex,
  showspaces=false,
  showstringspaces=false,
  aboveskip=0.5ex
}

\begin{document}

\section*{Lecture aims}

 \begin{enumerate}

 \item To critique the difference between program proof and other forms of verification, such as testing.

 \item To be able to prove the correctness of simple programs against their contracts.

 \item To think about programming in a more systematic and mathematical manner, with the result of improving our programming skills.

 \end{enumerate}

\section*{Lecture plan}
  
\subsubsection*{Introduction to reasoning about correctness}

\begin{enumerate}

  \item Start with video: \url{http://www.youtube.com/watch?v=H61d_OkIht4}

  \item Nulka is a ``soft-kill'' decoy missile, which leads homing missiles away from ships to splash harmlessly in the water. Developed in Melbourne by BAE and DSTO.

  \item Nulk's flight control unit software formally verified using SPARK (at the programming level), using Hoare logic (topic for next 3-4 lectures).

 \item Now deployed on over 130 Australian, Canadian and United States warships, and is the most successful Australia defence export.

\end{enumerate}

\subsubsection*{Introduction to reasoning about programs}

\begin{enumerate}

 \item Semantics vs syntax. Three major types of semantics: 

 \begin{enumerate}

   \item denotational: translate programs into denotations; e.g. sets of legal traces; Dijkstra's weakest precondition semantics.

   \item operational: describes how programs are executed directly; e.g. as an abstract machine.

   \item axiomatic: describes programs by the axioms that hold over them; the meaning is what is provable; e.g. Hoare logic.

 \end{enumerate}

 \item We have specifications and designs (with contracts), but we need to implement our code such that it conforms to its contract.

 \item Testing, review, and ``basic'' static analysis are not enough --- many applications require \emph{proof} of correctness.

 \item Binary search example (below) --- where is the fault?

  First version of the binary search  appeared in 1946.  Not until 1962 that the first {\em correct} version appeared. Common version ``proved'' by John Bentley contains a fault!

 If the best and brightest computer scientists cannot produce a correct program, and faults can remain undetected in text books for decades, what hope do we have?

\end{enumerate}

\subsubsection*{A small programming language}

\begin{enumerate}

 \item Language has:

 \begin{enumerate}

  \item Procedures.

  \item Expressions: numbers, arithmetic, variables reference and array reference.

  \item Boolean expressions

  \item Statements: variable assignment, array assignment, ``skip'', sequencing, branching, iteration, and procedure calls.

 \end{enumerate}

 \item A \emph{program} is a list of procedures and a statement that calls those procedures.

\end{enumerate}

\subsubsection*{Hoare logic}

\begin{enumerate}

 \item Hoare triples.

 \item Inference rules.

 \item Example: increment an integer (below) --- can we prove this?

 \item The rules:

  \begin{enumerate}

   \item Assignment axiom. Example (Inc program): $\{x \geq 0\}~ x := x + 1~ \{x > 0\}$.
 
   Note that the rule at first seems backwards, but going forwards does not work using incremented example above.

   \item Consequence rule: Example: complete the Inc program proof. Additional example: $\{\textrm{true}\} ~x := 5~ \{x \geq 5\}$.

   \item Sequential composition rule: Example: swap program (below). 

    First, assume simultaneous assignment of $x$ and $y$. This motivates the use of \emph{auxiliary} variables and pre-state values.

    Then do sequential version, in which $t=x$\~{} cannot be proved. This motivates the sequential composition rule. Example: complete the swap program.

   \item Empty statement axiom. Example: see condition rule example.

   \item Conditional rule. Example: conditional swap (below)

   \item Iteration rule: Example: summation (below).

     Discuss the idea and challenge of loop invariants.

     Discuss the notion of partial correctness and termination. The following program is an example of 

\begin{algorithmic}[0]
\State $\{y \geq 0\}$
\While {$y \geq 0$}
  \State $x := 0;$
\EndWhile
\State $\{y \geq 0\}$
\end{algorithmic}

   Apply iteration rule to prove hypothesis: $\{y \geq 0 \} ~x := 0 \{ y \geq 0 \}$, which allows us to conclude:

\begin{algorithmic}[0]
\State $\{y \geq 0\}$
\While {$y \geq 0$}
  \State $x := 0;$
\EndWhile
\State $\{y \geq 0 \land \neg y \geq 0\}$
\end{algorithmic}

which has a contradiction. The program does not achieve this because it does not terminate, but even if it did terminate, it couldn't achieve that postcondition.

 \item Loop invariants. Requires some creativity, but is an active research area, even in the department.

  Properties of loop invariant $I$:

  \begin{enumerate}

    \item $P \implies I$: must be \emph{weaker} than its precondition.

    \item  $I \land \neg B \implies Q$: \emph{stronger} than its postcondition.

    \item $\{B \land I\} ~S~ \{I\}$: loop re-establishes the invariant.

\end{enumerate}

 Of course, these assume the loop body is correct!


  Some heuristics:

  \begin{enumerate}

    \item Loop invariants generally contain most of the variables from the loop condition, loop body, the precondition, and the postcondition (where we mean the precondition/postcondition of the loop, not necessarily the entire program).

    \item Generally state a relationship between the variables.

    \item Generally contain a predicate similar to the postcondition.

    \item Hold even if the loop condition is false.

  \end{enumerate}

 \end{enumerate}

 \item Array assignment rule: Example: $ \{ x = y \land a[x] = 0\}  ~a[y] := 5~ \{ a[x] = 0\}$.

   Initial attempt at rule: $\{P[E/a[NE]]\}~a[NE] := E~ \{P\}$.

   But if $x=y$, then $a[x] = 5$, but we cannot tell from the program.

   Hoare's corrected rule: $\{P[a\{NE \rightarrow E\}/a]\}~a[NE] := E~ \{P\} $

  \item Procedure call. Example: invalid example below. Proper example: factorial call.

   Programs and calls must have the following properties:

  \begin{enumerate}

    \item The number of arguments in a procedure call must be equal to the number of parameters in the procedure.
      
    \item The formal parameter names must be distinct from each other.

    \item The arguments for value-result variables must be distinct from each other. It is this rule that prevents the problem outlined in the counterexample above --- the procedure call would be illegal.

    \item Value parameters cannot be changed in the procedure body.

    \item No recursion.
   
  \end{enumerate}

  Look at recursive procedures and the induction rule.

\end{enumerate}

\subsubsection*{Mechanisation}

\begin{enumerate}

 \item Annotate the program with the necessary predicates. Automatic for all except loop invariant.

 \item Use a program to generate \emph{verification conditions}. 

 \item Prove using theorem prover.

 \item Left over involve user input (but can still be automated).

 \item All dischared implies correct. One or more failed implies incorrect. One or more undischarged either way implies unknown.

\end{enumerate}

\subsection*{Dijkstra's weakest precondition semantics}


\begin{enumerate}

 \item Weakest precondition: For  statement $S$ and postcondition $R$, the weakest precondition, written $wp(S, R)$, is the weakest predicate that establishes $R$ if $S$ is executed and terminates.

 \item The transformation rules: show table.

 \item Note that $\{ wp(S, Q) \}~ S ~\{Q\}$.

 \item To prove $\{P\} ~S~ \{Q\}$, need to just prove $P \rightarrow wp(S, Q)$.

\end{enumerate}


\pagebreak

\lstset{language=Java}
\lstset{aboveskip=3mm}
\lstinputlisting{\rootdir/hoare-logic/code/BinarySearch.java}

\vfill

\lstset{language=Ada}
\lstset{aboveskip=3mm}
\begin{lstlisting}
  procedure Inc (X: in out Integer)
     Pre => (X >= 0),
     Post => (X > 0);
  is begin
     X := X + 1;
  end Inc;
\end{lstlisting}
 
\vfill


\lstset{aboveskip=3mm}
\begin{lstlisting}
  procedure Swap (X, Y : in out Float)
  with
      Post => (X = Y'Old and Y = X'Old);
  is 
      T : Float;
  begin
      T := X;
      X := Y; 
      Y := T;
  end Swap;
\end{lstlisting}

\vfill 

\pagebreak

\lstset{aboveskip=3mm}
\begin{lstlisting}
    procedure ConditionalSwap (X, Y : in out Float)
    with
        Post => (X >= Y);
    is 
        T : Float;
    begin
        if X < Y then
            T := X;
            X := Y; 
            Y := T;
        endif;
    end Swap;
\end{lstlisting}

\vfill


\lstset{aboveskip=3mm}
\begin{lstlisting}[escapeinside={||}]
   procedure Summation (N : in Integer; A : in Array; Sum : out Integer)
   with
     Pre => (N >= 0),
     Post => (Sum = |$\sum_{j=0}^{N-1}$| A(J));
   is begin
      I := 0;
      Sum := 0;
      while I < N do
         Sum := Sum + A[I];
         I := I + 1;
      done;
   end Summation;
\end{lstlisting}

\vfill
\begin{displaymath}
  \begin{array}{lllll}
\textrm{Assume:} & p(x, y) \sdef x := y + 1 & & (1)\\[2mm]
                 & \{y + 1 = y + 1\}~ x := y + 1~ \{x = y + 1\} & \textrm{[Assignment axiom]} & (2)\\[2mm]
                 & true ~\implies ~ y + 1 = y + 1 & \textrm{[Logical theorem]} & (3)\\[2mm]
 \textrm{From 2, 3:} & \{true\}~ x := y + 1~ \{x = y + 1\} & \textrm{[Consequence rule]} & (4)\\[2mm]
 \textrm{From 1, 4:} & \{true\}~ p(x,y)~ \{ x = y + 1\} & \textrm{[Procedure rule]} & (5)\\[2mm]
 \textrm{From 5:}    & \{true\}~ p(z,z)~ \{ z = z + 1\} & \textrm{[Procedure rule]} & (6)
\end{array}
\end{displaymath}

\vfill

\pagebreak

\vfill

\begin{verbatim}
   procedure Factorial (N : in Natural; F : out Natural)
   with
     Post => (F = N!)
   is begin


      F : = 1;


      I : = 0;


      while I /= N do


         I := I + 1;



         F := F * I;



      done;



   {F = N!}
   end Factorial;

\end{verbatim}
\vfill


\pagebreak

\vfill

\begin{verbatim}
   procedure InsertionSort (A : in out Array)
   with
     Post => (for all m in (0 .. A'Length-2) => A(J) < A(J + 1));
   is begin


      while I /= N - 2 do



         J := I;


         while J /= 1 and A[J - 1] > A[J] do
     


             swap(A, J - 1, J);


             J := J - 1;


         done;


      done;


   {for all m in (0 .. A'Length-2) => A(J) < A(J + 1)}
   end Summation;

\end{verbatim}
\vfill

\pagebreak

\begin{tabular}{ll}
\\[4mm]
  $\{P[E/x]\}~ x := E~ \{P\}$ & (assignment axiom)\\[10mm]
$\{P\}~ \textbf{skip}~ \{P\}$ & (empty statement axiom)\\[10mm]
 $\begin{array}{c}
   P' \implies P,~  Q \implies Q', ~\{P\} ~S~ \{Q\}\\
 \hline
 \{P'\} ~S~ \{Q'\}
 \end{array}$ & (consequent rule)\\[10mm]
 $\begin{array}{c}
  \{P\} ~S_1~ \{R\},~~ \{R\} ~S_2~ \{Q\} \\
 \hline
 \{P\} ~S_1; S_2~ \{Q\}
 \end{array}$ & (sequential composition rule)\\[10mm]
$\begin{array}{c}
  \{P \land B\} ~S_1~ \{Q\},~~ \{P \land \neg B\} ~S_2~ \{Q\} \\
 \hline
 \{P\}~ \textbf{if}~ B ~\textbf{then}~ S_1 ~ \textbf{else}~ S_2 ~ \textbf{endif}~\{Q\}
 \end{array}$ & (conditional rule)\\[10mm]
 $\begin{array}{c}
  \{P \land B\} ~S~~\{P\}  \\
 \hline
 \{P\}~ \textbf{while}~ B ~\textbf{do}~ S ~ \textbf{done}~\{\neg B \land P\}
 \end{array}$ & (iteration rule)\\[10mm]
 $\begin{array}{c}
  \{P\}~S~ \{Q\}\\
\overline{\{P  [E_1/v_1, E_n/v_n]\} ~p(E_1, \ldots, E_n)~ \{Q [E_1/v_1, \ldots, E_n/v_n]\}}
 \end{array}$ & (procedure call rule)\\[10mm]
 $\{P[a\{NE \rightarrow E\}/a]\}~a[NE] := E~ \{P\}$ & (array assignment rule)\\[10mm]
\end{tabular}

\pagebreak
\vspace{2mm}

\begin{tabular}{llll}
\multicolumn{3}{c}{\textbf{Dijkstra's predicate transformers}}\\
\toprule
\textbf{Rule} & \textbf{Input} & \textbf{Output}\\
\midrule
Skip &       $wp(\algorithmicskip, R)$  & $R$\\[2mm]
Assignment  & $wp(x := E, R)$ & $R[E/x]$\\[2mm]
Sequence    & $wp(S_1; S_2, R)$ & $wp(S_1, wp(S_2, R))$\\[2mm]
Conditional & $wp(\algorithmicif~B~\algorithmicthen~S_1 ~\algorithmicelse~ S_2 ~\algorithmicend\algorithmicif, R)$ & $B \rightarrow wp(S_1, R) \land \neg B \rightarrow wp(S_2, R)$\\[2mm]
While loop  & $wp(\algorithmicwhile~ B ~\algorithmicdo~ S ~\algorithmicdone)$
            & $\exists k . (k \geq 0 \land P_k)$\\
          & & where $P_0 \equiv \neg B \land Q$\\
          & & \quad\quad $P_{k+1} \equiv B \land wp(S, P_k)$\\
\bottomrule
\end{tabular}


\end{document}  


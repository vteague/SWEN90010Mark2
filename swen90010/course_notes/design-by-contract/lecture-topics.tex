\documentclass[11pt]{article}

\def\rootdir{../}

\usepackage{fullpage}
\usepackage{booktabs}
\usepackage{listings}
\usepackage{color}

\definecolor{javared}{rgb}{0.6,0,0} % for strings
\definecolor{javagreen}{rgb}{0.25,0.5,0.35} % comments
\definecolor{javapurple}{rgb}{0.5,0,0.35} % keywords
\definecolor{javadocblue}{rgb}{0.25,0.35,0.75} % javadoc
\definecolor{javabackground}{rgb}{0.9,0.9,0.9}
\definecolor{javablack}{rgb}{0,0,0}


\lstset{language=Ada,
  backgroundcolor=\color{javabackground},
  basicstyle=\ttfamily\fontsize{10}{12}\selectfont,
  keywordstyle=\color{javablack}\bfseries,
  aboveskip={1.5\baselineskip},
  stringstyle=\color{javared},
  commentstyle=\color{javagreen},
  morecomment=[s][\color{javadocblue}]{/**}{*/},
  numbers=left,
  numberstyle=\tiny\color{black},
  frame=single,
  numbersep=10pt,
  stepnumber=1,
  tabsize=8,
  xleftmargin=0ex,
  xrightmargin=0ex,
  showspaces=false,
  showstringspaces=false,
  aboveskip=0.5ex
}



\begin{document}

\section*{Lecture aims}

 \begin{enumerate}

 \item To understand the motivation, advantages, and disadvantages of design by contract.

 \item To specify formal contracts for small- and medium-scale packages in SPARK.


 \end{enumerate}

\section*{Lecture plan}
  
\subsubsection*{Introduction to DbC}

\begin{enumerate}

  \item Based on the metaphor of ``clients'' and ``suppliers'', who have \emph{obligations} for and receive  \emph{benefits} from (respectively) a mutual \emph{contract}. Contracts are formal, mathematical statements specifying module behaviour.

  \item Developed by Bertrand Meyer, while a professor University of California, Santa Barbara. DbC + Eiffel efforts resulted in the 2006 ACM Software System Award.

  \item Re-iterate concepts of information hiding, encapsulation, and ADTs.

\end{enumerate}

\subsubsection*{Specifying ADT interfaces}

\begin{enumerate}

 \item Importance of interfaces (for information hiding of ADTs).
 
 \item Discuss comments as interface specifications, and show example below.

 \item Discuss idea of preconditions and postconditions as \emph{structured comments}, and show 2nd example below.

 \item Advantages of pre/post approach: (1) obligations are clear; and (2) forces designer to consider issues (especially preconditions and postconditions).

 \item Introduce the concept of a \emph{contract}:  an agreement between two or more parties with mutual obligations and benefits.  Show first table below.

 \item Metaphor of contracts and software: module is the supplier and the calling code is the client. Using contracts on interfaces ensures that interaction between components is governed by contracts.

 \item Contracts have at least: invariant, precondition, and postcondition.

\end{enumerate}

\subsubsection*{Formal contracts}

\begin{enumerate}

 \item DbC paradigm advocates that software designers should define \emph{formal}, \emph{precise} and \emph{verifiable} contracts. 

 \item Draw similarities with \emph{assertions}. The DbC paradigm goes further:  assertions are so valuable in establishing software correctness, they should be a routine part of the software design.

 \item Show Java  example below.

 \item Advantages of formal contracts: precise behaviour (documentation) --- thinking over the issues goes a long way towards ensuring correctness; runtime assertions; test oracles; static analysis; and no duplication of precondition checks.  Disadvantages: small overhead, which is recovered quickly.

\end{enumerate}

\subsubsection*{SPARK contracts}

\begin{enumerate}

 \item Pre/post annotations are optional (unlike ``own'' and ``derives'').

 \item SPARK pre/post annotations are propositions + quantifiers. All normal SPARK/Ada expressions can be used as well.

  \item Show Swap\_Add\_Max example below. Stress importance of preconditions for Add and Divide operations

 \item Things to note: \texttt{V'Old} to refer to pre-state value; similarity to using prime in Alloy; contracts specify a state machine.

\end{enumerate}

\pagebreak

\lstset{language=Java}
\lstset{aboveskip=3mm}
\begin{lstlisting}
//Implements a bounded stack data type.
//The stack must also be less than a specified size.
public class StackImpl implements IStack
{
  //Create a new empty stack, with a maximum size, which
  // must be greater than 0
  public StackImpl(int size);

  //Add Object o to the top of the stack
  // if the stack is not full.
  public void push(Object o);

  //Return and remove the object at the top of the stack
  // if the stack is not empty.
  public Object pop();

  //Return the size of the stack.
  public int size();
}
\end{lstlisting}

\vfill

\begin{center}
\begin{tabular}{lll}
\toprule
   & \textbf{Obligation} & \textbf{Benefit}\\
\midrule
 \textbf{Client} & Pay for product & Receives product \\
 \textbf{Supplier}  & Provide product & Receives income\\
\bottomrule
\end{tabular}
\end{center}

\vfill

\begin{center}
\begin{tabular}{lll}
\toprule
   & \textbf{Obligation} & \textbf{Benefit}\\
\midrule
 \textbf{Client} & Establish precondition & Receives result of computation \\
 \textbf{Supplier}  & Provide result of computation & Has precondition established\\
\bottomrule
\end{tabular}
\end{center}

\vfill

\pagebreak

\lstset{language=Ada}
\lstset{aboveskip=3mm}
\lstinputlisting{\rootdir/design-by-contract/code/Swap_Add_Max.ads}

\pagebreak

\lstset{language=Ada}
\lstset{aboveskip=3mm}
\lstinputlisting{\rootdir/design-by-contract/code/linear_search.ads}


\end{document}  

% LocalWords:  init PreFill PostFill FillOK GSM de Franche

\section{Introduction to Cryptography and computer security}

%-------------------------------------------------------------------------
\subsection*{Aims}

The aim of this brief introduction to cryptography and electronic security is to introduce some of the fundamental tools of cryptography and show what goes wrong if they're not used properly.  We will concentrate on \emph{digital signatures} and their use in authentication.

{\bf Lecture 1:} An introduction to public key crypto and digital signatures.

Go to the University library online and download ``Understanding Cryptography: A Textbook for Students and Practitioners'' by 
Christof Paar and Jan Pelzl.

Before the first lecture please read 
\begin{itemize}
\item 1.4.1 (p.13), on modular arithmetic, and
\item 7.2 (p. 174-5), introducing RSA. 
\end{itemize}

During the lecture we will cover the first part of Ch10, p. 259 to 268., up to (but not including) ``RSA padding: the probabilistic signature standard (PSS).''
   
{\bf Optional } other reading (non-examinable, just for interest if you already know some cryptography): 
\begin{itemize}
\item Chapter 6 (public key crypto), 
\item Chapter 7 (RSA), 
\item Ch 10 (the rest; more about signatures).
\end{itemize}

{\bf Lecture 2:} will be a Guest lecture from Prof Alex Halderman at the University of Michigan, based on his paper on mining for vulnerable RSA keys.  Download the paper from 

{\small
\verb| https://www.usenix.org/system/files/conference/usenixsecurity12/sec12-final228.pdf|. }
\documentclass[a4paper,11pt]{article}

\usepackage{fullpage}

\begin{document}

\vspace{10ex}
\begin{center}
\textbf{\sc The University of Melbourne}\\[0.5ex]
\textbf{\sc SWEN90010: High Integrity Systems Engineering}\\[1ex]
%\textbf{\sc Second Semester, 2008}\\[1ex]
\textbf{\sc Learning Outcomes}
\end{center}
\vspace{5ex}

\noindent
By the end of this subject, a student should be able to do
the following (categorised by chapters in the course notes):

\subsection*{Chapter 1 --- An Introduction to HISE}

\begin{itemize}
 \item Define the term ``high-integrity system''
 \item Define the different classes of high-integrity system
\end{itemize}

\subsection*{Chapter 2 --- Ada}


\begin{itemize}
 \item Describe the features of Ada that make it suitable for high-integrity software
 \item Read and modify basic Ada programs
\end{itemize}

\subsection*{Chapter 3 --- Safety engineering}

\begin{itemize}
 \item Explain the role of safety engineering in the system engineering lifecycle.
 \item Discuss the role of accidents and incidents in the safety analysis process
 \item Perform a preliminary hazard analsyis using the HAZOP method
 \item Apply the fault-tree analysis method to a system for a given fault
\end{itemize}

\subsection*{Chapter 4 --- Model-based specification}

\begin{itemize}
 \item Explain the advantages and disadvantages of formal model-based specification in software engineering
 \item Apply basic logic, set/relational theory concepts to software-based problems
 \item Model a domain using the Alloy language
 \item Define and check assertions using the Alloy language and tool
\end{itemize}

\subsection*{Chapter 5 --- Fault-tolerant design}

\begin{itemize}
 \item Explain the concept of fault tolerance in systems engineering
 \item Compare hardware and software fault tolerance
 \item Design, analyse, and critique a fault-tolerant hardware design
 \item Design, analyse, and critique a fault-tolerant software design
 \item Implement algorithms for majority voting, median voting, and k-plurality voting
 \item Compare and contrast the different voting algorithms, and evaluate their use in specific systems
 \item Apply the concepts of  duplication, parity coding and checksums to small information redundancy problems, and explain the situations  in which cryptographic mechanisms are necessary instead.
 \item Analyse and explain the lower bounds for Byzantine Agreement in the authenticated and unauthenticated models, and the protocols that achieve those bounds.
\end{itemize}

\subsection*{Chapter 6-7 --- SPARK}

\begin{itemize}
  \item Describe the features of SPARK that make it suitable for high-integrity software
 \item Read and write basic SPARK programs
 \item Read and write basic preconditions and postconditions for SPARK programs.
\end{itemize}

\subsection*{Chapter 8 --- Reasoning about program correctness}

\begin{itemize}
 \item Explain the advantages and disadvantages of program proof compared to other program verification techniques.
 \item By hand, prove the correctness or otherwise of small programs using  Hoare logic 
\end{itemize}


\subsection*{Chapter X --- Security and cryptography}

\begin{itemize}

 \item Informally describe  the main security properties of digital signatures,  message authentication codes, and cryptographic hash functions.
 \item  Understand and analyse the logical trust structure of digital certificates, and synthesise a model of it in Alloy.
 \item  Recall and explain some specific recent examples of security problems caused by the use of weak or poorly implemented cryptography.
 \item Explain at a high level how protocols for Internet voting and digital cash use cryptography to achieve certain properties.

\end{itemize}


\end{document}
